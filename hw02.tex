\documentclass[11pt]{article}
%% Packages
\usepackage[T1]{fontenc}
\usepackage{amssymb}
\usepackage{amsmath}
\usepackage{amsthm}
\usepackage{amsfonts}
\usepackage{mathtools}
\usepackage{xspace}
\usepackage{bm}
\usepackage{nicefrac}
\usepackage{commath}
\usepackage{bbm}
\usepackage{boxedminipage}
\usepackage{xparse}
\usepackage{xcolor}
\usepackage{float}
\usepackage{multirow}
\usepackage{graphicx}
\usepackage{caption}
\usepackage{subcaption}
\usepackage{ifthen}
\usepackage{algpseudocode}
\usepackage{proba} 
\usepackage{colortbl} 
%\usepackage{upgreek}
%\usepackage{paralist}
%\usepackage{enumitem}
%\usepackage{times}

\usepackage{tikz}
\usetikzlibrary{positioning}
\usetikzlibrary{fit}
\usetikzlibrary{calc}
\usetikzlibrary{backgrounds}
\usetikzlibrary{shapes}
\usetikzlibrary{patterns}
\usetikzlibrary{matrix}

\usepackage{geometry}

%\usepackage[pdftex,bookmarks=true,pdfstartview=FitH,colorlinks,linkcolor=blue,filecolor=blue,citecolor=blue,urlcolor=blue]{hyperref}
%    \urlstyle{sf}

%\usepackage{enumitem}


%% Document Design Choices
%\setlength{\parindent}{0pt}
%\setlength{\parskip}{10pt}
%\renewcommand{\labelitemi}{\ensuremath{\circ}}
%\geometry{left=1in, right=1in, top=1in, bottom=1in}


%% Marking
\newcommand\TODO[1]{{\color{red}{#1}}\xspace}
\newcommand\change[1]{{\color{red}\underline{#1}}\xspace}
\newcommand{\commentdg}[1]{{\color{blue}{#1}}\xspace}
\newcommand\commentop[1]{{\color{red}{#1}}\xspace}


%% Theorems and Environments
\newtheorem{conjecture}{Conjecture}
%\newtheorem{theorem}{Theorem}
%\newtheorem{importedtheorem}{Imported Theorem}
%\newtheorem{informaltheorem}{Informal Theorem}
%\newtheorem{definition}{Definition}
%\newtheorem{lemma}{Lemma}
\newtheorem{claim}{Claim}
%\newtheorem{corollary}[theorem]{Corollary}
%\newtheorem{observation}{Observation}
%\newtheorem{property}{Property}


\newenvironment{boxedalgo}
  {\begin{center}\begin{boxedminipage}{1\textwidth}}
  {\end{boxedminipage}\end{center}}

\newenvironment{inlinealgo}[1]
  {\noindent\hrulefill\\\noindent{\bf {#1}}}
  {\hrulefill}


%% References
\newcommand{\namedref}[2]{\hyperref[#2]{#1~\ref*{#2}}\xspace}
\newcommand{\lemmaref}[1]{\namedref{Lemma}{lem:#1}}
\newcommand{\theoremref}[1]{\namedref{Theorem}{thm:#1}}
\newcommand{\inftheoremref}[1]{\namedref{Informal Theorem}{infthm:#1}}
\newcommand{\claimref}[1]{\namedref{Claim}{clm:#1}}
\newcommand{\corollaryref}[1]{\namedref{Corollary}{corol:#1}}
\newcommand{\figureref}[1]{\namedref{Figure}{fig:#1}}
\newcommand{\equationref}[1]{\namedref{Equation}{eq:#1}}
\newcommand{\defref}[1]{\namedref{Definition}{def:#1}}
\newcommand{\observationref}[1]{\namedref{Observation}{obs:#1}}
\newcommand{\procedureref}[1]{\namedref{Procedure}{proc:#1}}
\newcommand{\importedtheoremref}[1]{\namedref{Imported Theorem}{impthm:#1}}
\newcommand{\informaltheoremref}[1]{\namedref{Informal Theorem}{infthm:#1}}
\newcommand{\sectionref}[1]{\namedref{Section}{sec:#1}}
\newcommand{\appendixref}[1]{\namedref{Appendix}{app:#1}}
\newcommand{\propertyref}[1]{\namedref{Property}{prop:#1}}


%% English
\newcommand{\ie}{\text{i.e.}\xspace}
\newcommand{\st}{\text{s.t.}\xspace}
\newcommand{\etal}{\text{et al.}\xspace}
\newcommand{\naive}{\text{na\"ive}\xspace}
\newcommand{\wrt}{\text{w.r.t.}\xspace}
\newcommand{\whp}{\text{w.h.p.}\xspace}
\newcommand{\resp}{\text{resp.}\xspace}
\newcommand{\eg}{\text{e.g.}\xspace}
\newcommand{\cf}{\text{cf.}\xspace}
\newcommand{\visavis}{\text{vis-\`a-vis}\xspace}

%% Authors
\newcommand{\damgard}{Damg{\aa}rd\xspace}


%% Crypto
\def\alice{{{Alice}}\xspace}
\def\bob{{{Bob}}\xspace}
\def\eve{{{Eve}}\xspace}
\newcommand{\poly}{\ensuremath{\mathrm{poly}}\xspace}
\newcommand{\polylog}{\ensuremath{\mathrm{polylog}\;}\xspace}
\newcommand{\negl}{\ensuremath{\mathsf{negl}}\xspace}
\newcommand{\secpar}{\ensuremath{{\lambda}}\xspace}
\newcommand{\zo}{\ensuremath{{\{0,1\}}}\xspace}

  %% Functionalities
  \newcommand{\func}[1]{\ensuremath{\cF_{\mathsf{#1}}}\xspace}
  \newcommand{\fcom}[0]{\func{com}}
  \newcommand{\fot}[0]{\func{ot}}
  
  %% Protocols
  % \newcommand{\prot}[1]{\ensuremath{\pi_{\mathsf{#1}}}\xspace}


%% Graph Theory
\newcommand{\ancest}[0]{\mathsf{Ancestors}}
\newcommand{\sibling}[0]{\mathsf{Siblings}}
\newcommand{\parent}[0]{\mathsf{parent}}
\newcommand{\leaves}[0]{\mathsf{leaves}}


%% Math Symbols
\renewcommand{\o}[0]{\ensuremath{\circ}\xspace}
\newcommand{\dist}[1]{\ensuremath{\left\langle{#1}\right\rangle}\xspace}
%\newcommand{\prob}[2]{\ensuremath{\dist{{#1}_1, \dotsc, {#1}_{#2}}}\xspace}
\newcommand{\ip}[2]{\ensuremath{\left\langle{#1},{#2}\right\rangle}\xspace}
\renewcommand{\vec}[1]{\ensuremath{\mathbf{#1}}\xspace}
\newcommand{\concat}[0]{\ensuremath{\circ}\xspace}
\newcommand{\nin}[0]{\ensuremath{\not\in}\xspace}
\newcommand{\xor}[0]{\ensuremath{\oplus}\xspace}
\newcommand{\rv}[1]{\ensuremath{\mathbf{#1}}\xspace}
\newcommand{\p}[1]{\ensuremath{^{{\left(#1\right)}}}\xspace}
\newcommand{\argmax}[0]{\ensuremath{\mathop{\mathrm{argmax}}~}\xspace}
\newcommand{\argmin}[0]{\ensuremath{\mathop{\mathrm{argmin}}~}\xspace}
\newcommand{\me}{\mathrm{e}}
\NewDocumentCommand\mathstack{>{\SplitList{;}}m}
  {\ensuremath{
    \begin{smallmatrix}
      \ProcessList{#1}{ \insertone }    
    \end{smallmatrix}
  }}
\newcommand{\insertone}[1]{\ensuremath{#1}\\}
\newcommand{\tuple}[1]{\ensuremath{{\left\langle{#1}\right\rangle}}\xspace}

%\newcommand{\argmax}{\operatornamewithlimits{argmax}}
%\DeclareMathOperator*{\argmax}{arg\,max}

  
  %% General
  \newcommand{\ceil}[1]{\ensuremath{\left\lceil{#1}\right\rceil}\xspace}
  \newcommand{\floor}[1]{\ensuremath{\left\lfloor{#1}\right\rfloor}\xspace}
  %\newcommand{\abs}[1]{\ensuremath{\left\vert{#1}\right\vert}\xspace}
  \newcommand{\lone}[1]{\ensuremath{\left\vert{#1}\right\vert}\xspace}
  \newcommand{\spnorm}[1]{\ensuremath{\left\Vert{#1}\right\Vert}\xspace}

  %% Renamed Symbols
  \newcommand{\eps}[0]{\varepsilon}
  \let\epsilon\eps
  %\let\phi\varphi
  
  %% Cal Alphabets
  \newcommand{\cA}{\ensuremath{{\mathcal A}}\xspace}
  \newcommand{\cB}{\ensuremath{{\mathcal B}}\xspace}
  \newcommand{\cC}{\ensuremath{{\mathcal C}}\xspace}
  \newcommand{\cD}{\ensuremath{{\mathcal D}}\xspace}
  \newcommand{\cE}{\ensuremath{{\mathcal E}}\xspace}
  \newcommand{\cF}{\ensuremath{{\mathcal F}}\xspace}
  \newcommand{\cG}{\ensuremath{{\mathcal G}}\xspace}
  \newcommand{\cH}{\ensuremath{{\mathcal H}}\xspace}
  \newcommand{\cI}{\ensuremath{{\mathcal I}}\xspace}
  \newcommand{\cK}{\ensuremath{{\mathcal K}}\xspace}
  \newcommand{\cL}{\ensuremath{{\mathcal L}}\xspace}
  \newcommand{\cM}{\ensuremath{{\mathcal M}}\xspace}
  \newcommand{\cN}{\ensuremath{{\mathcal N}}\xspace}
  \newcommand{\cO}{\ensuremath{{\mathcal O}}\xspace}
  \newcommand{\cP}{\ensuremath{{\mathcal P}}\xspace}
  \newcommand{\cQ}{\ensuremath{{\mathcal Q}}\xspace}
  \newcommand{\cR}{\ensuremath{{\mathcal R}}\xspace}
  \newcommand{\cS}{\ensuremath{{\mathcal S}}\xspace}
  \newcommand{\cT}{\ensuremath{{\mathcal T}}\xspace}
  \newcommand{\cU}{\ensuremath{{\mathcal U}}\xspace}
  \newcommand{\cV}{\ensuremath{{\mathcal V}}\xspace}
  \newcommand{\cW}{\ensuremath{{\mathcal W}}\xspace}
  \newcommand{\cX}{\ensuremath{{\mathcal X}}\xspace}
  \newcommand{\cY}{\ensuremath{{\mathcal Y}}\xspace}
  \newcommand{\cZ}{\ensuremath{{\mathcal Z}}\xspace}
  
  %% Bold Alphabets
  \newcommand{\bA}{\ensuremath{{\mathbf A}}\xspace}
  \newcommand{\bB}{\ensuremath{{\mathbf B}}\xspace}
  \newcommand{\bC}{\ensuremath{{\mathbf C}}\xspace}
  \newcommand{\bD}{\ensuremath{{\mathbf D}}\xspace}
  \newcommand{\bE}{\ensuremath{{\mathbf E}}\xspace}
  \newcommand{\bF}{\ensuremath{{\mathbf F}}\xspace}
  \newcommand{\bG}{\ensuremath{{\mathbf G}}\xspace}
  \newcommand{\bQ}{\ensuremath{{\mathbf Q}}\xspace}
  \newcommand{\bU}{\ensuremath{{\mathbf U}}\xspace}
  \newcommand{\bV}{\ensuremath{{\mathbf V}}\xspace}
  \newcommand{\bX}{\ensuremath{{\mathbf X}}\xspace}
  \newcommand{\bY}{\ensuremath{{\mathbf Y}}\xspace}
  \newcommand{\br}{\ensuremath{{\mathbf r}}\xspace}
  
  %% Black-board Bold Alphabets
  \newcommand{\bbA}{\ensuremath{{\mathbb A}}\xspace}
  \newcommand{\bbB}{\ensuremath{{\mathbb B}}\xspace}
  \newcommand{\bbC}{\ensuremath{{\mathbb C}}\xspace}
  \newcommand{\bbD}{\ensuremath{{\mathbb D}}\xspace}
  \newcommand{\bbE}{\ensuremath{{\mathbb E}}\xspace}
  \newcommand{\bbF}{\ensuremath{{\mathbb F}}\xspace}
  \newcommand{\bbG}{\ensuremath{{\mathbb G}}\xspace}
  \newcommand{\bbH}{\ensuremath{{\mathbb H}}\xspace}
  \newcommand{\bbI}{\ensuremath{{\mathbb I}}\xspace}
  \newcommand{\bbJ}{\ensuremath{{\mathbb J}}\xspace}
  \newcommand{\bbK}{\ensuremath{{\mathbb K}}\xspace}
  \newcommand{\bbN}{\ensuremath{{\mathbb N}}\xspace}
  \newcommand{\bbO}{\ensuremath{{\mathbb O}}\xspace}
  \newcommand{\bbP}{\ensuremath{{\mathbb P}}\xspace}
  \newcommand{\bbQ}{\ensuremath{{\mathbb Q}}\xspace}
  \newcommand{\bbR}{\ensuremath{{\mathbb R}}\xspace}
  \newcommand{\bbS}{\ensuremath{{\mathbb S}}\xspace}
  \newcommand{\bbT}{\ensuremath{{\mathbb T}}\xspace}
  \newcommand{\bbU}{\ensuremath{{\mathbb U}}\xspace}
  \newcommand{\bbV}{\ensuremath{{\mathbb V}}\xspace}
  \newcommand{\bbZ}{\ensuremath{{\mathbb Z}}\xspace}
  
  %% Fraktur Alphabets
  \newcommand{\fP}{\ensuremath{{\mathfrak P}}\xspace}
  \newcommand{\fR}{\ensuremath{{\mathfrak R}}\xspace}
  \newcommand{\fX}{\ensuremath{{\mathfrak X}}\xspace}
  
  %% Overline Aphabets
  \newcommand{\op}{\ensuremath{{\overline p}}\xspace}
  
  %% Hat Aphabets
  \newcommand{\hf}{\ensuremath{{\widehat f}}\xspace}
  \newcommand{\hg}{\ensuremath{{\widehat g}}\xspace}
  
  %% Fractions
  \newcommand{\half}{\ensuremath{\frac12}\xspace}
  
  %% Set
  \newcommand{\comp}[1]{\ensuremath{\overline{{#1}}}\xspace}
  
  %% Models
  \newcommand{\defeq}[0]{\ensuremath{{\;\vcentcolon=\;}}\xspace}
  \newcommand{\eqdef}[0]{\ensuremath{{\;=\vcentcolon\;}}\xspace}
  \newcommand{\entails}[0]{\ensuremath{{\;\models\;}}\xspace}
  
  %% Matrix
  \newcommand{\tran}[0]{\ensuremath{^{\mathsf{T}}}\xspace}
  
  %% Probability and Distributions
  \newcommand{\event}[1]{\ensuremath{\mathsf{#1}}\xspace}
  \newcommand{\supp}[0]{\ensuremath{\mathsf{Supp}}\xspace}
  \let\pr\prob
  %\newcommand{\pr}[0]{\mathop{\mathrm{Pr}}\xspace}
  \newcommand{\var}[0]{\mathop{\mathrm{Var}}\xspace}
  \newcommand{\cov}[0]{\mathop{\mathrm{Cov}}\xspace}
  %\renewcommand{\Pr}[0]{\mathrm{\generateerror}\xspace}
  %\newcommand{\E}[0]{\mathop{\bbE}\xspace}
  \newcommand{\getsr}[0]{\mathbin{\stackrel{\mbox{\,\tiny \$}}{\gets}}}
  \newcommand{\drawn}{\ensuremath{\sim}\xspace}
  %\newcommand{\sd}[2]{\ensuremath{\mathbf{\Delta}\left({#1},{#2}\right)}\xspace}
  \newcommand{\sd}[2]{\ensuremath{\mathrm{SD}\left({#1},{#2}\right)}\xspace}
  \newcommand{\hd}{\ensuremath{\mathrm{HD}}\xspace}
  \newcommand{\kl}[2]{\ensuremath{\mathrm{D_{KL}}\left({#1},{#2}\right)}\xspace}
  \newcommand{\iid}[0]{\text{i.i.d.}\xspace}
  \newcommand{\ent}[0]{\ensuremath{\mathrm{H}}\xspace}
  
    %% Random Variables
    \newcommand{\rvA}{\rv{A}}
    \newcommand{\rvB}{\rv{B}}
    \newcommand{\rvC}{\rv{C}}
    \newcommand{\rvD}{\rv{D}}
    \newcommand{\rvG}{\rv{G}}
    \newcommand{\rvP}{\rv{P}}
    \newcommand{\rvQ}{\rv{Q}}
    \newcommand{\rvR}{\rv{R}}
    \newcommand{\rvS}{\rv{S}}
    \newcommand{\rvT}{\rv{T}}
    \newcommand{\rvU}{\rv{U}}
    \newcommand{\rvV}{\rv{V}}
    \newcommand{\rvW}{\rv{W}}
    \newcommand{\rvX}{\rv{X}}
    \newcommand{\rvY}{\rv{Y}}
    \newcommand{\rvZ}{\rv{Z}}
    
    \newcommand{\rvm}{\rv{m}}
    \newcommand{\rvr}{\rv{r}}
    \newcommand{\rvx}{\rv{x}}
    \newcommand{\rvy}{\rv{y}}
    \newcommand{\rvc}{\rv{c}}

  %% Binary Operators
  \newcommand{\band}[0]{\ensuremath{~\wedge~}\xspace}
  \newcommand{\bor}[0]{\ensuremath{~\vee~}\xspace}
  \let\leq\leqslant
  \let\le\leqslant
  \let\geq\geqslant
  \let\ge\geqslant
  
  %% Combinatorial
  \renewcommand\choose[2]{\ensuremath{{
    \left(\begin{matrix}
      {#1}\\
      {#2}
    \end{matrix}\right)}}\xspace}
  \newcommand\smallchoose[2]{\ensuremath{{
    \left(\begin{smallmatrix}
      {#1}\\
      {#2}
    \end{smallmatrix}\right)}}\xspace}
  \newcommand{\fallfact}[2]{\ensuremath{{#1}_{\left({#2}\right)}}\xspace}
  %\newcommand{\fact}[1]{\ensuremath{{#1}!}\xspace}
  
  %% Set Operations
  \newcommand{\union}[0]{\ensuremath{\cup}\xspace}
  \newcommand{\intersect}[0]{\ensuremath{\cap}\xspace}
  \newcommand{\setdiff}[0]{\ensuremath{\Delta}\xspace}
  

%% Algorithms, Predicates
\newcommand{\pred}[1]{\ensuremath{\mathsf{#1}}\xspace}


%% Local Terms
\newcommand{\mycite}[1]{{\color{brown}{{#1}}}\xspace}
\newcommand{\wt}[0]{\ensuremath{\mathsf{wt}}\xspace}
\newcommand{\Inf}[0]{\ensuremath{\pred{Inf}}\xspace}
%\newcommand{\Var}[0]{\ensuremath{\pred{Var}}\xspace}
\newcommand{\sk}[0]{\pred{sk}} 
\newcommand{\pk}[0]{\pred{pk}} 
\newcommand{\gen}[0]{\pred{Gen}} 
\newcommand{\enc}[0]{\pred{Enc}} 
\newcommand{\dec}[0]{\pred{Dec}} 
\newcommand{\SK}[0]{\pred{sk}} 
\newcommand{\capprox}[0]{\approx\p c}
\renewcommand{\tag}[0]{\pred{Tag}} 
\newcommand{\sign}[0]{\pred{Sign}} 
\newcommand{\ver}[0]{\pred{Ver}} 
\newcommand{\simu}[0]{\pred{Sim}} 
\newcommand{\ball}[0]{\pred{Ball}\xspace}
\newcommand{\vol}[0]{\pred{Vol}\xspace}


\newcommand{\pout}[0]{\ensuremath{\p{\text{out}}}\xspace}
\newcommand{\pin}[0]{\ensuremath{\p{\text{in}}}\xspace}

\newcommand{\Ext}[0]{\ensuremath{\pred{Ext}}\xspace}
\newcommand{\inv}[0]{\pred{inv}}



\usepackage{fancyhdr}   
\pagestyle{fancy}      
\lhead{}               
\rhead{Harris Christiansen (christih@purdue.edu)} %%% <-- YOUR NAME HERE

\usepackage[strict]{changepage}  
\newcommand{\nextoddpage}{\checkoddpage\ifoddpage{\ \newpage\ \newpage}\else{\ \newpage}\fi}  


\begin{document}

\title{Homework 2}

\date{}

\maketitle 

\thispagestyle{fancy}  
\pagestyle{fancy}      




\begin{enumerate}

%%% Problem 1 
\item {\bfseries Defining Multiplication over $\bbZ_{27}^*$.} 
  In the class, we had considered the group $(\bbZ_{26},+)$ to construct a one-time pad for one alphabet messages. 
  A few students were interested to define a group with 26 elements using a ``multiplication''-like operation. 
  This problem will assist you to define the $(\bbZ_{27}^*,\times)$ group. 
  
  Interpret $\bbZ_{27}^*$ as the set of all triplets $(a_0,a_1,a_2)$ such that $a_0,a_1,a_2 \in \bbZ_3$ and at least one of them is non-zero (you can think of the triplets as the ternary representation of the elements in $\bbZ_{27}^*$). 
  We shall equivalently interpret the element $(a_0,a_1,a_2)$ as the polynomial $a_0 + a_1X + a_2X^2$. 
  So, every element in $\bbZ_{27}^*$ has an associated non-zero polynomial of degree at most 2, and every non-zero polynomial of degree at most 2 has an element in $\bbZ_{27}^*$ associated with it. 
  
  The multiplication ($\times$ operator) of the element $(a_0,a_1,a_2)$ with the element $(b_0,b_1,b_2)$ is defined as the element corresponding to the polynomial
    $$(a_0 + a_1X + a_2X^2) \times (b_0 + b_1X + b_2X^2) \mod X^3 + 2X + 2$$
  According to this definition of the $\times$ operator, find
  \begin{itemize}
  \item (10 points) $(1,2,1) \times (2,2,1) $
  \item (15 points) the inverse of $(1,2,1)$.
  \end{itemize}
  {\bfseries Solution.} 
  \begin{itemize} 
  \item $(1,2,1) \times (2,2,1) $
   	\[(a_0 + a_1X + a_2X^2) \times (b_0 + b_1X + b_2X^2) \mod X^3 + 2X + 22\]
 	\[(1 + 2X + X^2) \times (2 + 2X + X^2) \mod X^3 + 2X + 2\]
 	\[2 + 4X +2X^2) + (2X + 4X^2 + 2X^3) + (X^2 + 2X^3 + X^4 \mod X^3 + 2X + 2\]
 	\[2 + 0X + X^2 + X^3 + X^4 \mod X^3 + 2X + 2\]
 	\[2 + 0X + X^2 - (2X+2) - (2X^2+2X) \mod X^3 + 2X + 2\]
 	\[0 + 2X + 2X^2 \mod X^3 + 2X + 2\]
 	\[(0, 2, 2)\]
  \item the inverse of $(1,2,1)$. 
 	\[(1 + 2X + X^2) \times (b_0 + b_1X + b_2X^2) \mod X^3 + 2X + 2\]
 	\[(b_0 + b_1X + b_2X^2) + (2b_0X + 2b_1X^2 + 2b_2X^3) + (b_0X^2 + b_1X^3 + b_2X^4) \mod X^3 + 2X + 2\]
 	\[b_0 + (b_1+2b_0)X + (b_2+2b_1+b_0)X^2 + (2b_2+b_1)X^3 + b_2X^4 \mod X^3 + 2X + 2\]
 	\[b_0 + (b_1+2b_0)X + (b_2+2b_1+b_0)X^2 - (2b_2+b_1)(2X+2) - b_2(2X^2+2X) \mod X^3 + 2X + 2\]
 	\[b_0 + (b_1+2b_0)X + (b_2+2b_1+b_0)X^2 - (b_2+2b_1) - (b_2+2b_1)X - (2b_2)X - (2b_2)X^2 \mod X^3 + 2X + 2\]
 	\[(b_0+b_1+2b_2) + (2b_0+2b_1)X + (2b_2+2b_1+b_0)X^2 \mod X^3 + 2X + 2\]
 	\[(0, 0, 0) \textrm{ when } b_0 = 2, b_1 = 2, b_2 = 1\]
 	\[(1, 2, 2)\]
  \end{itemize}

\nextoddpage 
%%% Problem 2 
\item {\bfseries One-time Pad for 3-Alphabet Words.} 
  We interpret $a,b,\dotsc,z$ as $0,1,\dotsc,25$. 
  We will work over the group $(\bbZ^3_{26},+)$, where $+$ is coordinate-wise integer-sum $\mod 26$. 
  For example, $abx + acd = ada$. 
  
  Now, consider the one-time pad encryption scheme over the group $(\bbZ^3_{26},+)$. 
  \begin{itemize}
  \item (12.5 points) What is the probability that the encryption of the message $cat$ is the cipher text $cat$?
  $$P(cipher="cat" given message="cat") = P(secretkey="aaa") = (1/26)^3 = 0.0000569$$
  \item (12.5 points) What is the probability that the encryption of the message $cat$ is the cipher text $dog$? 
  $$P(cipher="dog" given message="cat") = P(secretkey="bon") = (1/26)^3 = 0.0000569$$
  \end{itemize}  
  
  
\nextoddpage 
%%% Problem 3 
\item {\bfseries Left Identity and Left Inverse.} 
  Recall that when we defined a group $(G,\circ)$, we stated that there exists an element $e$ such that for all $x\in G$ we have $x\circ e=x$. 
  Note that $e$ is ``applied on $x$ from the right.'' 
  
  Similarly, for every $x\in G$, we are guaranteed that there exists $\inv(x)\in G$ such that $x\circ\inv(x)=e$. 
  Note that $\inv(x)$ is again ``applied to $x$ from the right.'' 
  
  Intuitively, we shall explore the following questions: (a) Is there an ``identity from the left?,'' and (b) Is there an ``inverse from the left?'' 
  
  We shall formalize and prove these results in this question. 
  \begin{itemize}
  \item (10 points) Prove that $e\circ x = x$, for all $x\in G$. 
  \item (10 points) Prove that if there exists an element $\alpha\in G$ such that for all $x\in G$ we have $\alpha\circ x=x$, then $\alpha = e$. 
  \end{itemize}
  Note that these two steps prove that the ``left identity'' is identical to the right identity $e$. 
  
  
  \begin{itemize}
  \item (10 points) Prove that $\inv(x)\circ x = e$. 
  \item (10 points) Prove that if there exists an element $\alpha\in G$ and $x\in G$ such that $\alpha\circ x=e$, then $\alpha=\inv(x)$. 
  \end{itemize}
  Note that these two steps prove that the ``left inverse of $x$'' is identical to the left inverse $\inv(x)$. 
  
  Finally, we can prove the following result crucial to the proof of security of one-time pad over the group $(G,\circ)$. 
  \begin{itemize}
  \item (10 points) Suppose $m\in G $ is a message and $c\in G$ is a cipher text. 
    Prove that there exists a unique $\sk\in G$ such that $m\circ \sk = c$. 
  \end{itemize}
  
  {\bfseries Solution.} 
  \begin{itemize} 
  \item {\bfseries Prove that $e\circ x = x$, for all $x\in G$.} \\
  Proof by contradiction. Suppose there exists $x$ such that $e \circ x \neq x$ \\
  By performing the $\circ \inv(x)$ function on both sides, we get $e \circ (x \circ inv(x)) \neq x \circ inv(x) \equiv e \circ e \neq e \equiv e \neq e $ \\
  Proof by contradiction, $e\circ x = x$
  
  \item {\bfseries Prove that if there exists an element $\alpha\in G$ such that for all $x\in G$ we have $\alpha\circ x=x$, then $\alpha = e$. } \\
  Suppose there exists $\alpha$ such that $\alpha \circ x = x$ \\
  Let $x = e$, such that $\alpha \circ e = e$ \
  Given $\alpha \circ e = \alpha$, we can conclude $\alpha = e$
  
  \item {\bfseries Prove that $\inv(x)\circ x = e$. } \\
  Given that $x \circ \inv(x) = e$ and  $x \circ e = x$ \\
  Then $\inv(x) = \inv(x) \circ e = \inv(x) \circ (x \circ \inv(x)) = (\inv(x) \circ x) \circ \inv(x) $ \\
  It follows that $e = \inv(x) \circ \inv(\inv(x)) = (\inv(x) \circ x) \circ \inv(x) \circ \inv(\inv(x)) = (\inv(x) \circ x) \circ e = \inv(x) \circ x$ \\
  $\inv(x) \circ x = e$
  
  \item {\bfseries Prove that if there exists an element $\alpha\in G$ and $x\in G$ such that $\alpha\circ x=e$, then $\alpha=\inv(x)$. } \\
  Suppose $\alpha \circ x = e$ \\
  Then  $\alpha \circ x = e \equiv \alpha \circ x \circ \inv(x) = e \circ \inv(x) \equiv \alpha \circ e = e \circ \inv(x) \equiv \alpha = \inv(x)$
  
  \item {\bfseries Suppose $m\in G $ is a message and $c\in G$ is a cipher text. 
    Prove that there exists a unique $\sk\in G$ such that $m\circ \sk = c$. } \\
  Given $m,c\in G $, suppose $m\circ \sk = c$ \\
  This would imply that $m\circ \sk \circ \inv(m) = c \circ \inv(m) \equiv \sk = c \circ \inv(m)$ \\
  Therefor, there exists a $\sk\in G$ for every $m,c\in G $ where $\sk = c \circ \inv(m)$
  \end{itemize}

\nextoddpage 
%%% Problem 4 
\item {\bfseries One-time Pad with non-uniform secret key.} 
  (25 points) Consider the one-time pad encryption scheme over a group $(G,+)$. 
  Suppose the a priori distribution of messages is the uniform distribution over the set $G$. 
  Suppose the generation algorithm samples the secret-key $\sk$ according to the distribution $\cD$ over the sample space $G$ such that $\cD$ is {\em not} the uniform distribution over $G$. 
  Is this encryption scheme secure? 
  
  (\footnotesize{\em Remark:} 
    To prove that the scheme is secure, provide a proof that the a priori distribution of messages is same as the a posteriori distribution. 
    To prove that the scheme is insecure, provide a proof that the a priori distribution of messages is different from the a posteriori distribution.)
    
    
    
\nextoddpage 
%%% Problem 5 
\item {\bfseries Designing Encryption Scheme.} 
  We shall work over the field $(\bbZ_{11},+,\times)$. 
  Assume that there are ten people $\{1,2,\dotsc,10\}$. 
  Design a private-key encryption scheme for the following scenario. 
  
  Alice meets the ten people $\{1,2,\dotsc,10\}$ today. 
  She can provide each of them information $\{s_1,s_2,\dotsc,s_{10}\}$. 
  
  Tomorrow, Alice shall encrypt a message $m\in\bbZ_{11}$. 
  The encryption has to ensure that decryption should be possible if and only if two people among $\{1,\dotsc,5\}$ and three people among $\{6,\dotsc,10\}$ get together.
  
  
  \begin{itemize}
  \item (15 points) Provide the $(\gen,\enc,\dec)$ algorithms. 
  \item (15 points) Proof of security of this scheme. 
  \end{itemize} 






\nextoddpage 
%%% Problem 6
\item {\bfseries A property of 2-wise Independence.} 
  Let \cH be a hash function family from the domain \cD to the range \cR. 
  \begin{itemize}
  \item (20 points) Similar to the proof in the lectures for universal hash function family, prove the following. 
    There exists distinct $x_1^*,x_2^*\in\cD$ and $y_1^*,y_2^*\in\cR$ such that 
      $$\probX{h(x^*_1)=y^*_1,h(x^*_2)=y^*_2\colon h\getsr\cH} \geq \frac1{\abs\cR^2}$$
    ({\em Remark:} Note that this result does not depend on whether $\abs\cR < \abs\cD$ or not.) 
  \item (25 points) Now, suppose that $\abs\cR < \abs\cD$. 
    Suppose that for all distinct $x_1,x_2\in\cD$ the following holds.  
    $$\probX{h(x_1)=h(x_2)\colon h\getsr\cH} < \frac1{\abs\cR}$$
    Prove that there exists distinct $x_1^*,x_2^*\in\cD$ and $y_1^*,y_2^*\in\cR$ such that 
      $$\probX{h(x^*_1)=y^*_1,h(x^*_2)=y^*_2\colon h\getsr\cH} > \frac1{\abs\cR^2}$$
  \end{itemize}
  This result proves that if a universal hash-function family has collision probability $<\frac1{\abs\cR}$ then it is not pairwise independent. 



\nextoddpage 
%%% Problem 7 
\item {\bfseries Extra Credit.} 
  Suppose $\cD=\zo^n$ and $\cR=\zo^{n-1}$. 
  Construct a hash function family such that for all distinct $x_1,x_2\in\cD$ we have 
  $$\probX{h(x_1)=h(x_2)\colon h\getsr\cH} = \frac1M \cdot\left(\frac{N-M}{N-1}\right),$$
  where $N=2^n$ and $M=2^{n-1}$.  
  Try to construct a hash function family such that each hash function can be efficiently evaluated. 
 



\end{enumerate}









\end{document}
